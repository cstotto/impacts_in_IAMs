\usepackage{xspace} 
\newcommand{\LArrow}{{\Large $\tc{hervor}{\Rightarrow}$~}}
\newcommand{\LASpace}{\\\hspace*{1.6em}}
\providecommand{\herv}[1]{\tc{hervor}{#1}}
\providecommand{\on}[1]{\operatorname{#1}}
\newcommand{\todo}[1]{{\color{red!80!black} TODO: {\itshape\color{orange!70!black} #1}}}
\DeclareRobustCommand{\acclimate}{\emph{acclimate}\xspace}
\DeclareRobustCommand{\Acclimate}{\emph{Acclimate}\xspace}

\usepackage{xspace}  %leerzeichen bei bedarf, aber nicht vor satzeichen
\providecommand{\tc}[2]{\textcolor{#1}{#2}}
\usepackage[]{units}
\providecommand{\myunit}[1]{$\,$\unit{#1}}
\providecommand{\myunitfrac}[2]{$\,\frac{\unit{#1}}{\unit{#2}}$}

%Mathoperators
\newcommand{\E}{\mathcal{E}}
\newcommand{\T}{\mathcal{T}}
\renewcommand{\P}{\mathcal{P}}
\newcommand{\Var}{\operatorname{Var}}
\newcommand{\DDt}{\operatorname{D_{\Delta t}}}
\newcommand{\tDDt}{\operatorname{\tilde{D}_{\Delta t}}}

\newcommand{\Cov}{\operatorname{Var}}
\newcommand{\mbR}{\mathbb{R}}
\newcommand{\mcF}{\mathcal{F}}
%\newcommand{\argmin}{\operatornamewithlimits{argmin}}
\newcommand{\argmax}{\operatornamewithlimits{argmax}}
\renewcommand{\minof}[2]{\min\left[#1\,,\,#2\right]}
\newcommand{\minofthree}[3]{\min\left[#1\,,\,#2\,,\,#3\right]}
\renewcommand{\maxof}[2]{\max\left[#1\,,\,#2\right]}
\newcommand{\pf}[2]{\frac{\partial#1}{\partial#2}}


\newcommand{\first}{1^{\operatorname{st}}}
\newcommand{\secnd}{2^{\operatorname{nd}}}
\newcommand{\third}{3^{\operatorname{rd}}}
\newcommand{\nth}{\text{n}^{\operatorname{th}}}
%\newcommand{\hPSD}{\hyperlink{PSD}{PSD}\xspace}

%##########################
% Real and imaginary parts
%##########################
\renewcommand{\Re}{\operatorname{Re}}
\renewcommand{\Im}{\operatorname{Im}}
%#############################
% Quantities
%#############################
\newcommand{\DA}{D^A}
\newcommand{\DK}{D^K}
\newcommand{\DKA}{D^{K,A}}
\newcommand{\DL}{D^L}
\newcommand{\LA}{L^A}
\newcommand{\bA}{\beta^A}
\newcommand{\dA}{\delta^A}
\newcommand{\dK}{\delta^K}
\newcommand{\GDP}{\operatorname{GDP}}
\newcommand{\gN}{g^{0}}
\newcommand{\gK}{g^{K}}
\newcommand{\gT}{g^{T}}
\newcommand{\AN}{A^{0}}
\newcommand{\tdK}{\tilde{\delta}^K}
\newcommand{\tdA}{\tilde{\delta}^A}

%#############################
% SCC
%#############################
\newcommand{\SCC}{\operatorname{SCC}}
\newcommand{\Lglob}{\mathcal{L}^{\operatorname{glob}}}
\newcommand{\Lreg}{\mathcal{L}^{\operatorname{reg}}}
\newcommand{\mcL}{\mathcal{L}}



\newcommand{\newleftside}{\ifthenelse{\isodd{\thepage}}{\newpage}{\newpage\phantom{placeholder}\thispagestyle{empty} \newpage}}
\newcommand{\newrightside}{\ifthenelse{\isodd{\thepage}}{\newpage\phantom{placeholder}\thispagestyle{empty} \newpage}{\newpage}}
%
\DeclareCaptionLabelFormat{cont}{#1~#2~(cont.)}

\newcommand{\define}{:=}
\newcommand{\set}{:=}
\newcommand{\from}{\leftarrow}
\renewcommand{\to}{\rightarrow}
\newcommand{\bndollar}[1]{\$#1bn}

\makeatletter
\newcommand{\colorwave}[1][blue]{\bgroup \markoverwith{\lower3.5\p@\hbox{\sixly \textcolor{#1}{\char58}}}\ULon}
\font\sixly=lasy6
\makeatother
\newcommand{\todoline}{\colorwave[red]}

\newcommand{\captioncr}{{\protect\\}}
\AtBeginDocument{\renewcommand\appendixname{Appendix~}}
\def\UrlBreaks{\do\/\do-\/\do_}

\makeatletter
% for aligning in math modes:
\newcommand{\pushright}[1]{\ifmeasuring@#1\else\omit\hfill$\displaystyle#1$\fi\ignorespaces}
\newcommand{\pushcenter}[1]{\ifmeasuring@#1\else\omit\hfill$\displaystyle#1$\hfill\fi\ignorespaces}
\newcommand{\pushleft}[1]{\ifmeasuring@#1\else\omit$\displaystyle#1$\hfill\fi\ignorespaces}
% instead of eqnarray use:
% \begin{alignat}{4} a&&\pushcenter{=}&b \end{alignat}
\makeatother

\newcommand{\annot}[2][]{%
  \pdfannot width \linewidth height 2\baselineskip depth 0pt{%
    /Subtype/Text%
    /Open false
    /Name /Comment%
    /CA .4%
    /C [.3 .6 .9]%
    /T (\pdfescapestring{#1})%
    /Contents(\pdfescapestring{\detokenize{#2}})%
  }
}
\ifdefined\note\relax\else
\newcommand\note[1]{\marginnote{\begin{minipage}{0.8\marginparwidth}\footnotesize\raggedright{}#1\end{minipage}}}
\fi

%#############################
% Sets
%#############################
\newcommand{\mbC}{\mathbb{C}}
\newcommand{\mbN}{\mathbb{N}}
\newcommand{\mbDA}{\mathbb{D}^A}
%%%%%%%%%%%%%%%%
% Farben
%%%%%%%%%%%%%%%%
\definecolor{dunkelgrau}{rgb}{0.8,0.8,0.8}
\definecolor{hellgrau}{RGB}{204,216,226}
