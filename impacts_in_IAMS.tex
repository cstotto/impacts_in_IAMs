%\documentclass[review]{elsarticle} % larger line spacing for more easy review
\documentclass[preprint,3p,authoryear]{elsarticle}

\usepackage{lineno}
\modulolinenumbers[5]

%\journal{Journal of Economic Dynamics \& Control}
\makeatletter
\def\ps@pprintTitle{%
 \let\@oddhead\@empty
 \let\@evenhead\@empty
 \def\@oddfoot{\hfill\today}%
 \let\@evenfoot\@oddfoot}
\makeatother

%% `Elsevier LaTeX' style
% \bibpunct{[}{]}{,}{a}{}{;}
\biboptions{round,semicolon}
\bibliographystyle{elsarticle-harv}
\let\cite\citep
%%%%%%%%%%%%%%%%%%%%%%%

%\smartqed  % flush right qed marks, \eg at endof proof
\clubpenalty = 10000 
\widowpenalty = 10000
\displaywidowpenalty = 10000
%Additional packages
\usepackage{lmodern}
\usepackage{supertabular}
\usepackage{amsmath,amssymb}
\usepackage[utf8x]{inputenc}
\usepackage[ngerman,english]{babel}
\usepackage[usenames,dvipsnames]{xcolor}
\usepackage{graphicx} % For includefigure
\usepackage{import}
\usepackage[list=off]{caption}
\usepackage[position=top]{subfig}
%nicer tables -----------------
\usepackage{booktabs}
\newcommand{\ra}[1]{\renewcommand{\arraystretch}{#1}}
\usepackage{enumerate} % roman numbering in enumerations
\usepackage{multirow}
\usepackage{mathtools}
\mathtoolsset{showonlyrefs}
\usepackage{microtype}
\usepackage{rotating}
\usepackage{url}
\usepackage[colorlinks=true,linkcolor=Blue,linktoc=page,citecolor=Blue,urlcolor=BrickRed,breaklinks]{hyperref}
\usepackage{doi}
%\usepackage{breakurl}
\def\UrlBreaks{\do\/\do-\/\do_}
%\renewcommand*{\backref}[1]{}
%\renewcommand*{\backrefalt}[4]{%
%  \ifcase #1 %
%    (Not cited.)%
%  \or
%    (Cited on page~#2.)%
%  \else
%    (Cited on pages~#2.)%
%  \fi}
%\renewcommand*{\backrefsep}{, }
%\renewcommand*{\backreftwosep}{ and~}
%\renewcommand*{\backreflastsep}{ and~}
\newcommand{\RNum}[1]{\uppercase\expandafter{\romannumeral #1\relax}}
\usepackage{xspace} 
\newcommand{\LArrow}{{\Large $\tc{hervor}{\Rightarrow}$~}}
\newcommand{\LASpace}{\\\hspace*{1.6em}}
\providecommand{\herv}[1]{\tc{hervor}{#1}}
\providecommand{\on}[1]{\operatorname{#1}}
\newcommand{\todo}[1]{{\color{red!80!black} TODO: {\itshape\color{orange!70!black} #1}}}
\DeclareRobustCommand{\acclimate}{\emph{acclimate}\xspace}
\DeclareRobustCommand{\Acclimate}{\emph{Acclimate}\xspace}

\usepackage{xspace}  %leerzeichen bei bedarf, aber nicht vor satzeichen
\providecommand{\tc}[2]{\textcolor{#1}{#2}}
\usepackage[]{units}
\providecommand{\myunit}[1]{$\,$\unit{#1}}
\providecommand{\myunitfrac}[2]{$\,\frac{\unit{#1}}{\unit{#2}}$}

%Mathoperators
\newcommand{\E}{\mathcal{E}}
\newcommand{\T}{\mathcal{T}}
\renewcommand{\P}{\mathcal{P}}
\newcommand{\Var}{\operatorname{Var}}
\newcommand{\DDt}{\operatorname{D_{\Delta t}}}
\newcommand{\tDDt}{\operatorname{\tilde{D}_{\Delta t}}}

\newcommand{\Cov}{\operatorname{Var}}
\newcommand{\mbR}{\mathbb{R}}
\newcommand{\mcF}{\mathcal{F}}
%\newcommand{\argmin}{\operatornamewithlimits{argmin}}
\newcommand{\argmax}{\operatornamewithlimits{argmax}}
\renewcommand{\minof}[2]{\min\left[#1\,,\,#2\right]}
\newcommand{\minofthree}[3]{\min\left[#1\,,\,#2\,,\,#3\right]}
\renewcommand{\maxof}[2]{\max\left[#1\,,\,#2\right]}
\newcommand{\pf}[2]{\frac{\partial#1}{\partial#2}}


\newcommand{\first}{1^{\operatorname{st}}}
\newcommand{\secnd}{2^{\operatorname{nd}}}
\newcommand{\third}{3^{\operatorname{rd}}}
\newcommand{\nth}{\text{n}^{\operatorname{th}}}
%\newcommand{\hPSD}{\hyperlink{PSD}{PSD}\xspace}

%##########################
% Real and imaginary parts
%##########################
\renewcommand{\Re}{\operatorname{Re}}
\renewcommand{\Im}{\operatorname{Im}}
%#############################
% Quantities
%#############################
\newcommand{\DA}{D^A}
\newcommand{\DK}{D^K}
\newcommand{\DKA}{D^{K,A}}
\newcommand{\DL}{D^L}
\newcommand{\LA}{L^A}
\newcommand{\bA}{\beta^A}
\newcommand{\dA}{\delta^A}
\newcommand{\dK}{\delta^K}
\newcommand{\GDP}{\operatorname{GDP}}
\newcommand{\gN}{g^{0}}
\newcommand{\gK}{g^{K}}
\newcommand{\AN}{A^{0}}
\newcommand{\tdK}{\tilde{\delta}^K}
\newcommand{\tdA}{\tilde{\delta}^A}

%#############################
% SCC
%#############################
\newcommand{\SCC}{\operatorname{SCC}}
\newcommand{\Lglob}{\mathcal{L}^{\operatorname{glob}}}
\newcommand{\Lreg}{\mathcal{L}^{\operatorname{reg}}}
\newcommand{\mcL}{\mathcal{L}}



\newcommand{\newleftside}{\ifthenelse{\isodd{\thepage}}{\newpage}{\newpage\phantom{placeholder}\thispagestyle{empty} \newpage}}
\newcommand{\newrightside}{\ifthenelse{\isodd{\thepage}}{\newpage\phantom{placeholder}\thispagestyle{empty} \newpage}{\newpage}}
%
\DeclareCaptionLabelFormat{cont}{#1~#2~(cont.)}

\newcommand{\define}{:=}
\newcommand{\set}{:=}
\newcommand{\from}{\leftarrow}
\renewcommand{\to}{\rightarrow}
\newcommand{\bndollar}[1]{\$#1bn}

\makeatletter
\newcommand{\colorwave}[1][blue]{\bgroup \markoverwith{\lower3.5\p@\hbox{\sixly \textcolor{#1}{\char58}}}\ULon}
\font\sixly=lasy6
\makeatother
\newcommand{\todoline}{\colorwave[red]}

\newcommand{\captioncr}{{\protect\\}}
\AtBeginDocument{\renewcommand\appendixname{Appendix~}}
\def\UrlBreaks{\do\/\do-\/\do_}

\makeatletter
% for aligning in math modes:
\newcommand{\pushright}[1]{\ifmeasuring@#1\else\omit\hfill$\displaystyle#1$\fi\ignorespaces}
\newcommand{\pushcenter}[1]{\ifmeasuring@#1\else\omit\hfill$\displaystyle#1$\hfill\fi\ignorespaces}
\newcommand{\pushleft}[1]{\ifmeasuring@#1\else\omit$\displaystyle#1$\hfill\fi\ignorespaces}
% instead of eqnarray use:
% \begin{alignat}{4} a&&\pushcenter{=}&b \end{alignat}
\makeatother

\newcommand{\annot}[2][]{%
  \pdfannot width \linewidth height 2\baselineskip depth 0pt{%
    /Subtype/Text%
    /Open false
    /Name /Comment%
    /CA .4%
    /C [.3 .6 .9]%
    /T (\pdfescapestring{#1})%
    /Contents(\pdfescapestring{\detokenize{#2}})%
  }
}
\ifdefined\note\relax\else
\newcommand\note[1]{\marginnote{\begin{minipage}{0.8\marginparwidth}\footnotesize\raggedright{}#1\end{minipage}}}
\fi

%#############################
% Sets
%#############################
\newcommand{\mbC}{\mathbb{C}}
\newcommand{\mbN}{\mathbb{N}}
\newcommand{\mbDA}{\mathbb{D}^A}
%%%%%%%%%%%%%%%%
% Farben
%%%%%%%%%%%%%%%%
\definecolor{dunkelgrau}{rgb}{0.8,0.8,0.8}
\definecolor{hellgrau}{RGB}{204,216,226}

\AtBeginDocument{\renewcommand\appendixname{Appendix~}}

%\input{latexdiff}

\begin{document}
%\showthe\columnwidth 338
%\date{\today}

\begin{frontmatter}

  \title{How to account for climate impacts in macroeconomic growth models?}

\author[PIK]{C.~Otto\corref{mycorrespondingauthor}}
\cortext[mycorrespondingauthor]{Corresponding author}
% \ead{christian.otto@pik-potsdam.de}


\author[PIK]{F.~Piontek}
\author[PIK]{B.~Sörgel}
\author[PIK]{T.~Geiger}
\author[PIK]{K.~Frieler}


\address[PIK]{Potsdam Institute for Climate Impact Research, Potsdam, Germany}

% \begin{abstract}
% \end{abstract}

% \begin{keyword}
%   disaster impact analysis\sep
%   higher-order effects\sep
%   economic network\sep
%   resilience\sep
%   dynamic input-output model\sep
%   agent-based modeling
% %Q540 Climate; Natural Disasters and Their Management; Global Warming
% %F100 Trade 
% %F180 Trade and Environment
% %Q560 Environment and Development; Environment and Trade; Sustainability; Environmental Accounts and Accounting; Environmental Equity; Population Growth
% %\JEL F100\sep F180\sep Q540\sep Q560
% \end{keyword}


\end{frontmatter}

%\linenumbers

%\newpage

\section{Integrated assessment models}
\label{sec:set}
The production system in integrated assessment models (IAMs) is often described by an aggregate production function of constant-elasticity of substition type \cite{BAU17} 
\begin{equation}
  \label{eq:CD}
  Y(\tau,T)\equiv \DA(T)A(\tau)\left[\xi_K(\theta_K \DK(T)K)^{\rho_Y}+\xi_L(\DL(T)\theta_LL)^{\rho_Y}+\xi_E\theta_\E\E)^{\rho_Y}\right]^{\frac{1}{\rho_Y}}
\end{equation}
where $\tau$ and $T$ are the slow timescale of the IAM (e.g, decadal) and the global mean temperature, respectively. Further, $\xi_K$, $\xi_L$, and $\xi_E$ denote shares of the production factors capital $K$, labor $L$, and $energy$, respectively, and $\theta_K$, $\theta_L$, and $\theta_E$ denote capital, labor and energy efficiencies. The intertemporal elasticity of substitution is denoted by $\rho_Y$. Damages to total factor productivity $A$, capital stock$ K$, and labor productivity $\theta_L$ are denoted by $\DA$, $\DK$, and $\DL$, respectively. They are only implicitly time dependent via the global mean temperature $T$. 
\begin{itemize}
\item \todo{So far, the capital stock appears to be the only impact channel for which it make sense to derive impact channel specific losses, because capital shock can trigger investment decisions.}
\end{itemize}
\section{Empirical analysis}
\label{sec:empAna}
In this section, we discuss how the damage terms $\DK$, $\DA$, and as well as losses to the GDP growth rate can be derived by regressing economic data with hazard indicators derived from the ISMIP impact model simulations. Since, we aim to apply these damages to different IAM realizations, i.e., different scenarios for capital accumulations, etc., it is key to consider relative damages only.

\subsection{GDP growth rate losses}
\label{subsec:g}
Following \citet{BAK19} and \citet{HSI14}, we use panel regressions of the form
\begin{subequations}
\label{eq:PLGDP}
\begin{align}
  g_{j,t}\equiv \ln(\GDP_{j,t})-\ln(\GDP_{j,t-1}) &= \gamma_j + \delta_{t} + \theta_j t + \sum_{l=0}^{L}\beta_{l}P_{j,t-l}+\epsilon_{j,t}\notag\\
  &= \gN_{j,t}+\delta_{j,t}
\end{align}
\end{subequations}
to capture the effects of climate extremes on county-level growth domestic product $\GDP_{j,t}$, where index $j$ labels the countries, and all time-dependent observables are assumed to vary on the fast (discrete) timescale $t$ of the bio-physical impacts (e.g, annual or monthly).
In the above equation, the timeseries of $\GDP$ are de-trended log-linearly by accounting for country and year fixed-effects denoted by $\gamma_j$, and $\delta_t$, respectively, as well as for country-specific time trends $\theta_j t$. The persistent effects of climate extremes on $\GDP$ are measured by the coefficients $\beta_l$, where the index $l\in[0,\ldots,L]$ numbers the time-lags up to the maximum lag-time $L$. Further, the $P_{j,t-l}$'s denote the population affected, and standard errors are denoted by $\epsilon_{j,t}$. In the second line of Eq.~\eqref{eq:PLGDP}, we have rewritten the GDP growth rate $g_{r,t}$ at time $t$ as the sum of the growth rate $g_{j,t}^0\equiv\gamma_j+\delta_t+\theta_j t+\epsilon_{j,t}$ of a reference system that is not perturbed in the time span $t-L,\ldots,t$, and the cumulative growth rate reduction 
\begin{equation}
  \label{eq:delta}
  \delta_{j,t}\left(\lbrace P_{j,t-m}^i\rbrace_{m=0}^L\right)\equiv g_{j,t}\left(\lbrace P_{j,t-m}^i\rbrace_{l=0}^L\right)-g_{j,t}^0= \sum_{l=0}^L\beta_lP_{j,t-l}\quad\forall j,t,
\end{equation}
resulting from a series of hazards $\lbrace P_{j,t-m}^i\rbrace_{m=0}^L$ striking at timepoints  $t-L,\ldots,t$.

\begin{itemize}
\item \todo{People affected for each category may have to be rescaled by their variance or category dependent vulnerability factors may have to be introduced.}
\item \todo{The damage notation may be misleading, since $\delta_{j,t}$ are only country dependent due to the impact realisation. The loss coefficients $\beta_l$ are the same for all countries or groups of countries. However, we may apply the regressions to sub-groups of countries, e.g., developing countries or OECD countries as done in \cite{BER16a,BER18}.}
\end{itemize}
From Eqs.~\eqref{eq:PLGDP}, it follows that the logarithmic GDP path of the perturbed system may be written as
\begin{subequations}
  \label{eq:GDP}
\begin{align}
  \ln(\GDP_{j,t})&=\ln(\GDP_{j,t-1})+\gN_{j,t}+\sum_{i=0}^L\beta_iP_{j,t-i}\notag\\
  &= \ln(\GDP_{j,0})+\sum_{t'=0}^{t-1}\Big[\gN_{t-t'}+\underbrace{\sum_{i=0}^L\beta_iP_{j,t-t'-i}}_{=\delta_{j,t-t'}}\Big]=\sum_{\tilde{t}=1}^{t}\left[\gN_{j,\tilde{t}}+\delta_{j,\tilde{t}}\right]\notag\\
 \Leftrightarrow\qquad\qquad\qquad \langle g \rangle_t&\equiv\ln(\GDP_{j,t})-\ln(\GDP_{j,0})=\sum_{\tilde{t}=1}^{t}\left[\gN_{j,\tilde{t}}+\delta_{j,\tilde{t}}\right]\label{subeq:avGrowthRate},
\end{align}
\end{subequations}
where, in Eq.~\eqref{subeq:avGrowthRate}, we have introduced the temporal averaged GDP growth rate for the time period under consideration.
\subsection{Losses in total factor productivity}
\label{subsec:TFP}
Following \citet{BAK19}, we use panel regressions of the form
\begin{subequations}
\label{eq:PLTFP}
\begin{align}
\ln(A_{j,t}) &= \gamma^{A}_j + \dA_{t} + \theta^{A}_j t + \sum_{l=0}^{\LA}\bA_{l}P_{j,t-l}+\epsilon^{A}_{j,t}\notag\\
             &= \ln(\AN_{j,t})+
               \ln(\dA_{j,t})=\ln(\AN_{j,t}\dA_{j,t})\label{subeq:PLTFPb}
\end{align}
\end{subequations}
to estimate losses in in total factor productivity $A_{j,t}$ (or any other production factor on which extreme events are expected to have persistent impacts).
Analogous to Eq.~\eqref{eq:PLGDP}, the timeseries of $A_{j,t}$ are de-trended log-linearly by accounting for country and year fixed-effects denoted by $\gamma^{A}_j$, and $\delta^{A}_t$, respectively, as well as for country-specific time trends $\theta^{A}_j t$. The persistent effects of climate extremes on TFP are measured by the coefficients $\bA_l$, where the index $l\in[0,\ldots,L]$ numbers the time-lags up to the maximum lag-time $\LA$. Standard errors are denoted by $\epsilon^{A}_{j,t}$. In Eq.~\eqref{subeq:PLTFPb}, we have rewritten $\ln(A_{j,t})$  as the sum of the logarithmic TFP, $\ln(\AN_{j,t})\equiv\gamma^A_j+\delta^A_t+\theta^A_j t+\epsilon^A_{j,t}$, of a reference system that is not perturbed during the time span $t-L,\ldots,t$, and the cumulative reduction in logarithmic TFP
\begin{equation}
  \label{eq:delta}
  \ln\Big(\dA_{j,t}\left(\lbrace P_{j,t-m}^i\rbrace_{m=0}^{\LA}\right)\Big)\equiv \ln\Big(A_{j,t}\left(\lbrace P_{j,t-m}^i\rbrace_{l=0}^{\LA}\right)\Big)-\ln(\AN_{j,t})= \sum_{l=0}^{\LA}\bA_lP_{j,t-l}\quad\forall j,t,
\end{equation}
resulting from a series of hazards $\lbrace P_{j,t-m}^i\rbrace_{m=0}^L$ striking at timepoints  $t-L,\ldots,t$.

From Eq.~\eqref{subeq:PLTFPb}, we see that $A_{j,t}$ may be rewritten as
\begin{equation}
  \label{eq:TFP} A_{j,t}=\AN_{j,t}\dA_{j,t}=\AN_{j,t}\exp\left(\sum_{i=0}^{L}\bA_iP_{j,t-i}\right)\quad\forall j,t.
\end{equation}
\subsection{Losses to the capital stock}
\label{subsec:K}
Losses to the capital stock are assumed to be instantaneous; their persistencies are endogenously calculated by the macroeconomic growth model. They are implemented as relative shocks to the capital stock $K$,
\begin{equation}
  \label{eq:dk}
  \dK_{j.t}\equiv 1-\frac{\Delta K_{j.t}}{K_{j,t}}, 
\end{equation}
where $\Delta K_{j,t}\leq 0$, denotes the absolute shock $K_{j,t}$ at time $t$.

\section{From time to temperature dependent impacts}
The ISIMIP2b modeling round includes $2$ RCPs (RCP2.6,RCP6.0),  $4$ GCMs, $1$ SSPs (SSP2), and a various number $\mbN_c$ of global impact models for each event category $c\in\mbC$.
We expect that impacts scale with global mean temperature (determined by the RCP-GCM) combination and with socioeconomic development (determined by the SSP). However, there is a good chance that the temperature scaling is universal, and the dependence of relative impacts on the RCP and also the SSP may be negligible.  
Thus for each RCP-SSP combination, we obtain $4\times c$ impact timeseries for each sector $c$. We first group countries according to the world regions of $r$ the IAM. We then check for a possible scaling of the regional damages with global (annual?) mean temperature or with (regional/national annual?) mean temperature by, separately for each region, regressing the country level damages with mean temperature 
\begin{equation}
  \label{eq:deltaT}
  \delta_{j,t} = \gamma^T_j + \beta^T_0T_{r,t} + \beta^T_1 T^2_{r,t},
\end{equation}
where $\gamma_j^T$ capture country level fixed damage levels, and $\beta^T_0$ and $\beta^T_1$, denote the coefficients for the linear and quadratic temperature dependence, respectively. For simplicity, we have dropped the superscipts denoting the type of damages (e.g., GDP growth reduction, TFP reductions, and losses to the capital stock).

% \section{From time to temperature dependent impacts}
% Future ISIMIP impact simulations are driven by RCP-GCM scenario combinations. For a given socioeconomic development scenario, these define bijective mappings from time $t$  to global mean temperature $T\equiv T(t)$. Thus, we may fit the damages terms $\delta_{j,t}$, $\dK_{j,t}$, and $\dA_{j,t}$ in terms of $T$, to obtain smooth temperature dependent damage functions denoted by $\delta_{j,T},\dK_{j,T}$, and $\dA_{j,T}$.



%  employ these damages functions not only for the RCP-GCM determined temperatures paths they were derived from, but also for the temperature paths 

% Let$\Delta T\equiv T(t_f)-T(t_0)$ with $t_f>t_0$ denote the interval of global mean temperature of interest. We then subdivide $\Delta T$ into $N_b$ equidistant bins $\Delta T_b$ of width $\Delta T/N_b$. Since the mapping from time to global mean temperature is bijective, we obtain for each temperature bin and each realization $\lbrace E\rbrace_{j,t,d}$ the associated set of cumulative damages as


% The ISIMIP2b modeling round includes $1$ SSP (SSP2.6), $4$ GCMs and a various number $\mbN_c$ of global impact models for each event category $c\in\mbC$. For each RCP-SSP-GCM combination, we may obtain a unique temperature path providing a bijective mapping from time $t$ to global mean temperature . For each of these temperature paths, we may then derive $\mbN_d\equiv \prod_{c\in\mbC} \mbN_c$ realization of bio-physical hazards from the  ISIMIP simulations.


% \begin{equation}
%   \label{eq:DDA}
%   \mbDA_{j,d}(\Delta T_b)\equiv\lbrace \DA_{j,t,d}\, \vert\, t' \text{ for } T(t')\in\Delta T_b \rbrace.
% \end{equation}
% To obtain relative TFP reductions that can be plugged into the IAM production function (cf.~Eq.~\eqref{eq:CD}), we have to take ensemble averages over (i) the members of the set $\mbDA_{j,d}(\Delta T_d)$ and (ii) the different realizations of biophysical hazards $\mbN_d$,
% \begin{equation}
%   \label{eq:avDA}
%   \langle \DA_{j}\rangle(\Delta T_b)\equiv\big\langle\langle \DA_{j,t,d}\rangle_{\mbN_D}\big\rangle_{\mbDA_{j}}(\Delta T_b).
% \end{equation}
% where $\langle\cdot\rangle$ denotes the ensemble average.

% \begin{itemize}
% \item Does the order in which we take the averages matter?
% \end{itemize}

%\section{Averaging over different impact realization}
% where $\lbrace E\rbrace_{j,t,d}$ denotes a realization $d\in\mbN_d$ of bio-physical hazards for country $j$ up to time $t$ drawn from the ISIMIP impact ensemble. The ISIMIP2b modeling round includes $1$ SSP (SSP2.6), $4$ GCMs and a various number $\mbN_c$ of global impact models for each event category $c\in\mbC$. For each RCP-SSP-GCM combination, we may obtain a unique temperature path providing a bijective\footnote{\tc{red}{Is this true?}} mapping from time $t$ to global mean temperature $T\equiv T(t)$. For each of these temperature paths, we may then derive $\mbN_d\equiv \prod_{c\in\mbC} \mbN_c$ realization of bio-physical hazards from the  ISIMIP simulations.


% \section{Regional integration}
% \label{sec:regInt}
% Since most IAMs operate on the level of world regions $r\in\mbN_r$, we have to average Eq.~\eqref{eq:avDA} over the countries $j$ belonging to each region $r$ (weighted by average population or GDP or asset stock\footnote{\tc{red}{To be discussed!}})  in order to obtain damages $\langle \DA_{r}\rangle(\Delta T_b)$ for the time period associated with $\Delta T_b$.

\section{Social Cost of Carbon}
\label{sec:SCC}

\begin{itemize}
\item Modularization of economic system, impacts and energy system
\end{itemize}
An alternative approach to integrate climate impacts into IAMs is to first derive and analytical formula for the social cost of carbon $\SCC$ and soft couple it with the IAM, and the climate module.

\subsection{Semi-analytic expression for $\SCC$}
\label{subsec:analytSCC}
Let us assume that the each regional economy $r$ growth exponentially with rate $g_{r,t}+\delta_{r,t}$, where $g_{r,t}$ denotes the growth rate of the unperturbed growth path (without climate shocks) and $\delta_{r,t}$ describes climate induced deviations from this path. Regional output $Y_{r,t}$ then obeys the following linear difference equation
\begin{equation}
  \label{eq:expGrowth}
  Y_t\equiv\big[1+g_{r,t}+\delta_{r,t}\big]Y_{t-1}.
\end{equation}
The latter may be simplified by a coordinate transformation to a frame growing with the rate $g_{n,t}$ of the unperturbed system
\begin{equation}
  \label{eq:growFrame}
  \tilde{Y}_t\equiv\prod_{t'=0}^t\left[1+g_{t'}\right]^{-1}Y_{t'}
\end{equation}
Applying this coordinate transformation to Eq.~\eqref{eq:expGrowth} and dropping the regional index $n$ for simplicity, yields
\begin{align}
\prod_{t'=0}^t(1+g_{t'})\tilde{Y}_t &= \prod_{t'=0}^{t-1}\big(1+g_{t'}\big)\tilde{Y}_{t-1}\big(1+g_{t'}+\delta_{t'}\big)\notag\\                                   &=\prod_{t'=0}^{t}\big(1+g_{t'}\big)\tilde{Y}_{t-1}\left(1+\frac{\delta_{t'}}{1+g_{t'}}\right)\notag\\
  \Rightarrow \tilde{Y}_{t} &= \prod_{t'=0}^{t}\left(1+\frac{\delta_{t'}}{1+g_{t'}}\right)Y_0\label{eq:Ytilde},
\end{align}
where we have employed the relation $\tilde{Y}_0=Y_0$ in the last line. Transforming Eq.~\eqref{eq:Ytilde} back to the original coordinate system then yields
\begin{equation}
  \label{eq:1}
%  Y_t=D_tY^0_t,
Y_t=\underbrace{\prod_{t'=0}^{t}\left(1+\frac{\delta_{t'}}{1+g_{t'}}\right)}_{\equiv D_t}Y_t^0
\end{equation}
where
\begin{equation}
  \label{eq:D}
  D_t\equiv\prod_{t'=0}^{t}\left(1+\frac{\delta_{t'}}{1+g_{t'}}\right)
\end{equation}
and
\begin{equation}
  \label{eq:YNull}
 Y^0_t\equiv\prod_{t'=0}^{t} \big(1+g_{t'}\big)Y_0 
\end{equation}
 denote cumulative climate damages at time $t$ and the unperturbed growth path, respectively.

\subsubsection{Regional Langrangian for $\SCC$}
\label{sec:Lreg}
\begin{itemize}
\item Climate damages fully internal for regional planners
\item \todo{Is this a standard approach to derive the SCC?-> Ask Gunnar}
\end{itemize}
\begin{itemize}
\item Investments
\begin{equation}
  \label{eq:Ir}
  I_{r,t}\equiv K_{r,t+1}-(1-\delta_K)K_{r,t}
\end{equation}

\item Regional utility
\begin{equation}
  \label{eq:ur}
  u_{r,t}(c_{r,t})\equiv\sum_r \omega_r\frac{c_{r,t}^{1-\eta}-1}{1-\eta},
\end{equation}
where $\eta$ and $\omega_r$ denote the intertemporal elasticity of substitution and the regional Negishi weights, respectively.
\end{itemize}
\begin{multline}
  \label{eq:Lreg} \mcL_r(\T;E_r,c_r)\equiv\sum_{t'=0}^{T}\sum_{r'}\left[\omega_{r'}N_{r',t'}u\big(c_{r',t'}\big)(1+\rho)^{-t'}\right.\\
  \left.+\lambda_{r',t'}\left[Y_{r',t'}\big(E_{r',t'}\big)D_{r'}(T_{r',t'}) -c_{r',t'}N_{r',t'}-\left(K_{r',t'+1}-\big(1-\delta_K\big)K_{r',t'}\right)-p_{r,t}E_{r,t} \right]\right],
\end{multline}
where we have introduced the regional tax on carbon emissions $p_{r,t}$.
\begin{itemize}
\item Note that damages $D_{r,t}=D_{r,t}(T_{r,t})$ depend only implicitly via the regional temperature on time
\end{itemize}
The first order conditions then read
\begin{itemize}
\item Shadow price of consumption
  \begin{equation}
    \label{eq:dLdc}    \pf{L_r}{c_{r,t}}=N_{r,t}\omega_r(1+\rho)^{-t}c_{r,t}^{-\eta}-\lambda_{r,t}N_{r,t}\quad\Leftrightarrow\quad\lambda_{r,t} = \omega_r(1+\rho)^{-t}c_{r,t}^{-\eta}\quad\forall\,r,t
  \end{equation}
From Eq.~\eqref{eq:dLdc}, we may derive an expression for $\lambda_{r,t+1}$ in terms of $\lambda_{r,t}$,
\begin{equation}
  \label{eq:lambda_tpone} \lambda_{r,t-1}=\omega_r(1-\rho)^{1-t}c_{r,t-1}^{-\eta}=(1+\rho)\left(\frac{c_{r,t-1}}{c_{r,t}}\right)^{-\eta}\lambda_{r,t},
\end{equation}
which will permit us to simplify the golden rule of capital accumulation.
\item Golden rule for capital accumulation
  \begin{align} \pf{L_r}{K_{r,t}}=\lambda_{r,t}\left[D_{r,t}\pf{Y_{r,t}}{K_{r,t}}+1-\delta_K\right]-\lambda_{r,t-1}&= 0\notag\\
    \Leftrightarrow\quad 1+\underbrace{D_{r,t}\pf{Y_{r,t}}{K_{r,t}}-\delta_K}_{\equiv r_{r,t}} &= (1+\rho)\left(\frac{c_{r,t-1}}{c_{r,t}}\right)^{-\eta}\quad\forall\, r,t \label{eq:dLdK}
  \end{align}
\item Emission tax determines marginal value of emissions for production  
  \begin{equation}
    \label{eq:dLdE} \pf{L_r}{E_{r,t}}=\lambda_{r,t}\left[D_{r,t}\pf{Y_{r,t}}{E_{r,t}}-p_{r,t}\right]\quad\Leftrightarrow\quad p_{r,t}=D_{r,t}\pf{Y_{r,t}}{E_{r,t}} \quad\forall\,r,t 
  \end{equation}
\end{itemize}
Also, for deriving the social cost of carbon, we may note the following useful relations 
\begin{equation}
  \label{eq:Negishi}
  \omega_{r}\approx \frac{c_{r,t}^{\eta}}{N_{t}}\quad\text{with }N_t\equiv\sum_{r'}c_{r',t}^{\eta} \quad\forall\,r,t
\end{equation}
\begin{subequations}
  \label{eq:lambda}
\begin{align} \lambda_{r',t'}&=\frac{(1+\rho)^{-t'}}{N_t}\left(\frac{c_{r',t}}{c_{r',t'}}\right)^{\eta}\quad\forall\, r',t',t\label{subeq:delta_rptp}\\
  \overset{t'=t}{\Rightarrow}\quad
  \lambda_{r,t}&=\lambda_{t}=\frac{(1+\rho)^{-t}}{N_t}\quad \forall t\label{subeq:delta_rt}
\end{align}
\end{subequations}
\subsubsection{Global Langrangian for $\SCC$}
\label{sec:Lglob}
\begin{multline}
  \label{eq:Lglob} \mcL(\T;E_r,c_r)\equiv\sum_{t'=0}^{\T}\sum_{r'}\left[\omega_{r'}N_{r',t'}u\big(c_{r',t'}\big)(1+\rho)^{-t'}\right.\\
  \left.+\lambda_{r',t'}\left[Y_{r',t'}\big(E_{r',t'}\big)D_{r',t'}\big(T_{r',t'}\big) -c_{r',t'}N_{r',t'}-\left(K_{r',t'+1}-\big(1-\delta_K\big)K_{r',t'}\right) \right]\right]
\end{multline}
The first order conditions then read
\begin{align}
  \label{eq:partE}
  \frac{\partial\mcL}{\partial E_{r,t}} &= \lambda_{r,t}\underbrace{\frac{\partial Y_{r,t}}{\partial E_{r,t}}D_{r,t}}_{=p_{r,t}}+\sum_{t'=0}^{\T}\sum_{r'}\lambda_{r',t'}Y_{r',t'}\frac{\partial D_{r',t'}}{\partial E_{r,t}}=0\quad\forall\,r,t\notag\\
  \Rightarrow \quad p_{r,t} &= -\lambda_{r,t}^{-1}\sum_{t'=0}^{\T}\sum_{r'}\lambda_{r',t'}Y_{r',t'}\frac{\partial D_{r',t'}}{\partial E_{r,t}}\notag\\
                                              &= -\lambda_{r,t}^{-1}\sum_{t'=0}^{\T}\sum_{r'}\lambda_{r',t'}Y_{r',t'}\frac{\partial }{\partial E_{r,t}}\left[\prod_{t''=0}^{t'}\left[1+\frac{\delta_{r',t''}\left(\sum_{\tilde{t}=0}^{t''}E_{r',\tilde{t}}\right)}{1+g_{r,t''}} \right] \right]\notag\\
                                         &=-\lambda_{r,t}^{-1}\sum_{t'=t}^{\T}\sum_{r'}\lambda_{r',t'}Y_{r',t'} \sum_{t''=t}^{t'}\big(1+g_{n,t''}\big)^{-1}\pf{\delta_{r',t''}}{E_{r,t}}\prod_{\tilde{t}=0,\tilde{t}\neq t''}^{t'}\left[1+\frac{\delta_{r',\tilde{t}}}{1+g_{r,\tilde{t}}} \right]\notag\\
                                         &=-\lambda_{r,t}^{-1}\sum_{t'=t}^{\T}\sum_{r'}\lambda_{r',t'}Y_{r',t'}\underbrace{\prod_{\tilde{t}=0}^{t'}\left[1+\frac{\delta_{r',\tilde{t}}}{1+g_{r,\tilde{t}}}\right]}_{=D_{r',t'}}\sum_{t''=t}^{t'}\underbrace{(1+g_{r',t''}+\delta_{r',t''})^{-1}\pf{\delta_{r',t''}}{T_{r',t''}}}_{\equiv -\Theta_{r',t''}}\underbrace{\pf{T_{r',t''}}{E_{r,t}}}_{\equiv\Delta T_{r',t'',r,t}}\notag\\
&= \sum_{t'=t}^{\T}\sum_{r'}\Phi_{r',t',t}Y^0_{r',t'}D_{r',t'}\sum_{t''=t}^{t'}\Theta_{r',t''}\Delta T_{r',t'',r,t},  
\end{align}
% \begin{equation*}
%   \label{eq:2}
%   p_{r,t}= \sum_{t'=t}^{\T}\sum_{r'}\Phi_{r',t',t}Y^0_{r',t'}D_{r',t'}\sum_{t''=t}^{t'}\Theta_{r',t''}\Delta T_{r',t'',r,t},  
% \end{equation*}
where we have introduced the marginal change of the growth rate with temperature
\begin{equation}
  \label{eq:Theta}
  \Theta_{r,t}\equiv - (1+g_{r,t}+\delta_{r,t})^{-1}\pf{\delta_{r,t}}{T_{r,t}}
\end{equation}
as well as the marginal response of temperature in region response to emissions
\begin{equation}
  \label{eq:DeltaT}
  \Delta T_{r',t',r,t}\equiv\begin{cases}
    \pf{T_{r',t'}}{E_{r,t}} & \text{for}\quad t'\geq t,\\
    0 & \text{else}    .
  \end{cases}
\end{equation}
\begin{align}
  \label{eq:Phi} \Phi_{r',t',t}\equiv\lambda_{r,t}^{-1}\lambda_{r',t'}\overset{~\eqref{subeq:delta_rptp},~\eqref{subeq:delta_rt}}{=}(1+\rho)^{t-t'}\left(\frac{c_{r',t}}{c_{r',t'}}\right)^{\eta}&\overset{}{=} (1+\rho)^{-(t'-t)}\prod_{\tilde{t}=t+1}^{t'}\left(\frac{c_{r',\tilde{t}-1}}{c_{r',\tilde{t}}}\right)^{\eta} &\text{for }t\leq t'\\
       &\overset{\eqref{eq:dLdK}}{=}
         \begin{cases}
           1 &\text{for }t=t',\\
           \prod_{\tilde{t}=t+1}^{t'}(1+r_{r',\tilde{t}})^{-1} & \text{for }t<t'.
           \end{cases}
\end{align}

\begin{align*}
\Phi_{r',t',t}\equiv\lambda_{r,t}^{-1}\lambda_{r',t'}=
         \begin{cases}
           1 &\text{for }t=t',\\
           \prod_{\tilde{t}=t+1}^{t'}(1+r_{r',\tilde{t}})^{-1} & \text{for }t<t'.
           \end{cases}
\end{align*}


\begin{itemize}
\item \todo{SCC appear to decrease with increasing Damages}
\item \todo{Damages are not explicitly time dependent anymore, only indirect dependence via the regional temperature}
\end{itemize}

\subsection{Semi-analytic expression for $\SCC$ with capital losses}
\label{subsec:analytSCC}
Let us assume that in each region $r$, capital accumulation can be described by a simple Solow-Swan like equation
%\begin{subeqations}
  \begin{equation}
  \label{eq:Solow}
    K_{r,t+1} = (1-\delta_K)\dK_{r,t}K_{r,t}+\underbrace{s\AN_{r,t}\dK_{r,t}K_{r,t}}_{\equiv Y_{r,t}(K_{r,t})}
    = \left(1+\underbrace{s\AN_{r,t}-\delta_K}_{\equiv\gKN_{r,t}} +\tdK_{r,t}\right)K_{r,t},
  \end{equation}
  % \end{subeqations}
\begin{equation}
    K_{r,t+1} = (1-\delta_K)\dK_{r,t}K_{r,t}+\underbrace{s\AN_{r,t}\dK_{r,t}K_{r,t}}_{\equiv Y_{r,t}(K_{r,t})}
    = \left(1+\underbrace{s\AN_{r,t}-\delta_K}_{\equiv\gKN_{r,t}} +\tdK_{r,t}\right)K_{r,t},
  \end{equation}

  where we have assumed that the production function $Y_{r,t}(K_{r,t})\equiv s \AN_{r,t}\dK_{r,t}K_{r,t}$ is of $A-K$-type, where are $\AN_{r,t}$ describes TFP. Further, we have introduced  rescaled damage terms for damages to the capital stock, $\tdK_{r,t}\equiv(\dK_{r,t}-1)(1+\gKN_{r,t})$. This enables us to rewrite the equation for capital accumulation as a linear difference equation with growth rate  $\gKN_{r,t}+\tdK_{r,t}+\tdA_{r,t}$, where $\gKN_{r,t}\equiv s\AN_{r,t}-\delta_K $ denotes the growth rate of the unperturbed growth path (without climate shocks) and the terms $\tdK_{r,t}$ describe climate induced deviations from this path.

Analogous to the calculations in in Sec.~\ref{subsec:analytSCC}, we may solve Eq.~\eqref{eq:Solow} which permits to express the perturbed capital stock in terms of the unperturbed one,
\begin{equation}
  \label{eq:Solow2}
  K_{r,t}=\DKA_{r,t}K^0_{r,t}\quad\forall\,r,t,
\end{equation}
\begin{equation*} K_{r,t}=\underbrace{\prod_{t'=0}^{t}\left(1+\tdK_{r,t}\right)}_{\equiv\DKA_{r,t}}K^0_{r,t},
\end{equation*}

where
\begin{equation}
  \label{eq:D}
  \DKA_{r,t}\equiv\prod_{t'=0}^{t}\left(1+\tdK_{r,t}\right)
\end{equation}
and
\begin{equation}
  \label{eq:YNull}
 K^0_{r,t}\equiv\prod_{t'=0}^{t} \big(1+\gK_{r,t'}\big)K_{r,0} 
\end{equation}
 denote cumulative climate damages to the capital stock at time $t$ and the unperturbed growth of the capital stock, respectively.

 \subsubsection{Regional Langrangian for $\SCC$ with capital losses}
\label{sec:LKAreg}
From Eq.~\eqref{eq:Solow2}, we see that if the TFP $A_{r,t}$ is not affected by the climate shock and the production function is linear in $K_{r,t,}$ (e.g., of $A-K$ type), we may express the perturbed production as a product of a damage factor $\DK_{r,t}\equiv\prod_{t'=0}^{t}(1+\tdK_{r,t})$ and the GDP growth path of the unperturbed system $Y^{0}_{r,t}\equiv s A_{r,t}K_{r,t}^0$,
\begin{equation}
  \label{eq:GDP2}
  Y_{r,t}\equiv\DK_{r,t}Y^{0}_{r,t}.
\end{equation}

\begin{multline}
  \label{eq:LKAreg} \mcL_r(\T;E_r,c_r)\equiv\sum_{t'=0}^{T}\sum_{r'}\left[\omega_{r'}N_{r',t'}u\big(c_{r',t'}\big)(1+\rho)^{-t'}\right.\\
  \left.+\lambda_{r',t'}\left[\DK_{r',t'}Y_{r',t'}\big(E_{r',t'}\big) -c_{r',t'}N_{r',t'}-\left(\DK_{r',t'+1}K_{r',t'+1}-\big(1-\delta_K\big)\DK_{r',t'}K_{r',t'}\right)-p_{r,t}E_{r,t} \right]\right],
\end{multline}
where we have introduced the regional tax on carbon emissions $p_{r,t}$.
\begin{itemize}
\item Note that damages $\DK_{r,t}=\DK_{r,t}(T_{r,t})$ depend only implicitly on time via the regional temperature $T_{r,t}$
\end{itemize}
The first order conditions then read
\begin{itemize}
\item The Shadow price of consumption $\pf{L_r}{c_{r,t}}$ remains unchanged with respect to Eq.~\eqref{eq:dLdc}.

\item Golden rule for capital accumulation read
  \begin{align} \pf{L_r}{K_{r,t}}=\lambda_{r,t}\left[\DK_{r,t}\pf{(Y_{r,t})}{K_{r,t}}+(1-\delta_K)\DK_{r,t}\right]-\lambda_{r,t-1}\DK_{r,t-1}&= 0\notag\\
    \Leftrightarrow\quad \frac{\DK_{r,t}}{\DK_{r,t-1}}\left[1+\pf{Y_{r,t}}{K_{r,t}}-\delta_K\right] &= (1+\rho)\left(\frac{c_{r,t-1}}{c_{r,t}}\right)^{-\eta}\quad\forall\, r,t \label{eq:dLdK}
  \end{align}
\todo{we may simplify by assuming $\DK_{r,t-1}\approx\DK_{r,t}$ }
\item Emission tax determines marginal value of emissions for production  
  \begin{equation}
    \label{eq:dLdE} \pf{L_r}{E_{r,t}}=\lambda_{r,t}\left[\pf{Y_{r,t}}{E_{r,t}}-p_{r,t}\right]\quad\Leftrightarrow\quad p_{r,t}=\pf{Y_{r,t}}{E_{r,t}} \quad\forall\,r,t 
  \end{equation}
\end{itemize}
\todo{Here, it appears strange to assume that production depends on emission but not the capital stock if production is simply $Y= sAK$ }

Also, for deriving the social cost of carbon, we may note the following useful relations 
\begin{equation}
  \label{eq:Negishi}
  \omega_{r}\approx \frac{c_{r,t}^{\eta}}{N_{t}}\quad\text{with }N_t\equiv\sum_{r'}c_{r',t}^{\eta} \quad\forall\,r,t
\end{equation}
\begin{subequations}
  \label{eq:lambda}
\begin{align} \lambda_{r',t'}&=\frac{(1+\rho)^{-t'}}{N_t}\left(\frac{c_{r',t}}{c_{r',t'}}\right)^{\eta}\quad\forall\, r',t',t\label{subeq:delta_rptp}\\
  \overset{t'=t}{\Rightarrow}\quad
  \lambda_{r,t}&=\lambda_{t}=\frac{(1+\rho)^{-t}}{N_t}\quad \forall t\label{subeq:delta_rt}
\end{align}
\end{subequations}
\subsubsection{Global Langrangian for $\SCC$ with losses to the capital stock}
\label{sec:LKglob}
\begin{multline}
  \label{eq:LKglob} \mcL(\T;E_r,c_r)\equiv\sum_{t'=0}^{\T}\sum_{r'}\left[\omega_{r'}N_{r',t'}u\big(c_{r',t'}\big)(1+\rho)^{-t'}\right.\\
  \left.+\lambda_{r',t'}\left[\DK_{r',t'}Y_{r',t'}\big(E_{r',t'}\big) -c_{r',t'}N_{r',t'}-\left(\DK_{r',t'+1}\big(T_{r',t'+1}\big)K_{r',t'+1}-\big(1-\delta_K\big)\DK_{r',t'}\big(T_{r',t'}\big)K_{r',t'}\right) \right]\right]
\end{multline}
The first order conditions with respect to emissions then reads
% \begin{align}
%   \label{eq:partE}
%   \frac{\partial\mcL}{\partial E_{r,t}} &= \lambda_{r,t}\underbrace{\frac{\partial Y_{r,t}}{\partial E_{r,t}}}_{=p_{r,t}}+ \sum_{t'=0}^{\T}\sum_{r'}\lambda_{r',t'}\left[Y_{r',t'}\frac{\partial \DK_{r',t'}}{\partial E_{r,t}}-\pf{\DK_{r',t'+1}}{E_{r,t}}K_{r',t'+1}+(1-\delta_K)\pf{\DK_{r',t'}}{E_{r,t}}K_{r',t'}\right]=0\quad\forall\,r,t\notag\\
%   \Rightarrow \quad p_{r,t} &= -\lambda_{r,t}^{-1}\sum_{t'=0}^{\T}\sum_{r'}\lambda_{r',t'}\left[\left(Y_{r',t'}+(1-\delta_K)K_{r',t'}\right)\pf{\DK_{r',t'}}{ E_{r,t}}-K_{r',t'+1}\pf{\DK_{r',t'+1}}{E_{r,t}}\right]\notag\\
%                                         &=                                          -\lambda_{r,t}^{-1}\sum_{t'=0}^{\T}\sum_{r'}\lambda_{r',t'}\left[\left[Y_{r',t'}+(1-\delta_K)K_{r',t'}-K_{r',t'+1}(1+\gK_{r',t'+1}+\dK_{r',t'+1})\right]\pf{\DK_{r',t'}}{ E_{r,t}}-K_{r',t'+1}\DK_{r',t'}(1+\gK_{r',t'+1})^{-1}\pf{\dK_{r',t+1}}{E_{r,t}}\right]\notag\\
%                                         &=                                          -\lambda_{r,t}^{-1}\sum_{t'=0}^{\T}\sum_{r'}\lambda_{r',t'}Y_{r',t'}\frac{\partial }{\partial E_{r,t}}\left[\prod_{t''=0}^{t'}\left[1+\frac{\delta_{r',t''}\left(\sum_{\tilde{t}=0}^{t''}E_{r',\tilde{t}}\right)}{1+g_{r,t''}} \right] \right]\notag\\
%                                          &=-\lambda_{r,t}^{-1}\sum_{t'=t}^{\T}\sum_{r'}\lambda_{r',t'}Y_{r',t'} \sum_{t''=t}^{t'}\big(1+g_{n,t''}\big)^{-1}\pf{\delta_{r',t''}}{E_{r,t}}\prod_{\tilde{t}=0,\tilde{t}\neq t''}^{t'}\left[1+\frac{\delta_{r',\tilde{t}}}{1+g_{r,\tilde{t}}} \right]\notag\\
%                                          &=-\lambda_{r,t}^{-1}\sum_{t'=t}^{\T}\sum_{r'}\lambda_{r',t'}Y_{r',t'}\underbrace{\prod_{\tilde{t}=0}^{t'}\left[1+\frac{\delta_{r',\tilde{t}}}{1+g_{r,\tilde{t}}}\right]}_{=D_{r',t'}}\sum_{t''=t}^{t'}\underbrace{(1+g_{r',t''}+\delta_{r',t''})^{-1}\pf{\delta_{r',t''}}{T_{r',t''}}}_{\equiv -\Theta_{r',t''}}\underbrace{\pf{T_{r',t''}}{E_{r,t}}}_{\equiv\Delta T_{r',t'',r,t}}\notag\\
% &= \sum_{t'=t}^{\T}\sum_{r'}\Phi_{r',t',t}Y_{r',t'}D_{r',t'}\sum_{t''=t}^{t'}\Theta_{r',t''}\Delta T_{r',t'',r,t},  
% \end{align}
\begin{align}
  \label{eq:partE}
  \frac{\partial\mcL}{\partial E_{r,t}} &= \lambda_{r,t}\underbrace{\frac{\partial Y_{r,t}}{\partial E_{r,t}}}_{=p_{r,t}}+ \sum_{t'=0}^{\T}\sum_{r'}\lambda_{r',t'}\left[Y_{r',t'}\frac{\partial \DK_{r',t'}}{\partial E_{r,t}}-\pf{\DK_{r',t'+1}}{E_{r,t}}K_{r',t'+1}+(1-\delta_K)\pf{\DK_{r',t'}}{E_{r,t}}K_{r',t'}\right]=0\quad\forall\,r,t\notag\\
  \Rightarrow \quad p_{r,t} &= -\lambda_{r,t}^{-1}\sum_{t'=0}^{\T}\sum_{r'}\lambda_{r',t'}\left[\left(Y_{r',t'}+(1-\delta_K)K_{r',t'}\right)\pf{\DK_{r',t'}}{ E_{r,t}}-K_{r',t'+1}\pf{\DK_{r',t'+1}}{E_{r,t}}\right]\notag\\
                                        &=                                          -\lambda_{r,t}^{-1}\sum_{r'}\left[\sum_{t'=0}^{\T}\lambda_{r',t'}\left[Y_{r',t'}+(1-\delta_K)K_{r',t'}\right]\pf{\DK_{r',t'}}{ E_{r,t}}- \sum_{t'=0}^{\T} \lambda_{r',t'}K_{r',t'+1}\pf{\DK_{r',t+1}}{E_{r,t}}\right]\notag\\
                                        &=                                          \lambda_{r,t}^{-1}\sum_{r'}\left[\sum_{t'=t}^{\T}\lambda_{r',t'}\left[Y_{r',t'}+(1-\delta_K)K_{r',t'}\right]\DK_{r',t'}\sum_{\tilde{t}=t}^{t'}\Theta^{K}_{r',\tilde{t}}\Delta T_{r',\tilde{t},r,t}\right.\\                                        &\qquad\,\left.-\sum_{t'=t+1}^{\T}\lambda_{r',t'}K_{r',t'+1}\DK_{r',t'}\sum_{\tilde{t}=t+1}^{t'}\Theta^{K}_{r',\tilde{t}}\Delta T_{r',\tilde{t},r,t}\right]\notag\\
  &=    \lambda_{r,t}^{-1}\sum_{r'}\left[\sum_{t'=t}^{\T}\lambda_{r',t'}\left[Y^0_{r',t'}+(1-\delta_K)K^0_{r',t'}\right]\DK_{r',t'}\sum_{\tilde{t}=t}^{t'}\Theta^{K}_{r',\tilde{t}}\Delta T_{r',\tilde{t},r,t}\right.\\ &\qquad\,\left.-\sum_{t'=t}^{\T-1}\lambda_{r',t'+1}K^0_{r',t'}\DK_{r',t'+1}\sum_{\tilde{t}=t+1}^{t'-1}\Theta^{K}_{r',\tilde{t}}\Delta T_{r',\tilde{t},r,t}\right]\notag\\                                         %
% -\lambda_{r,t}^{-1}\sum_{t'=0}^{\T}\sum_{r'}\lambda_{r',t'}Y_{r',t'}\frac{\partial }{\partial E_{r,t}}\left[\prod_{t''=0}^{t'}\left[1+\frac{\delta_{r',t''}\left(\sum_{\tilde{t}=0}^{t''}E_{r',\tilde{t}}\right)}{1+g_{r,t''}} \right] \right]\notag\\
%                                          &=-\lambda_{r,t}^{-1}\sum_{t'=t}^{\T}\sum_{r'}\lambda_{r',t'}Y_{r',t'} \sum_{t''=t}^{t'}\big(1+g_{n,t''}\big)^{-1}\pf{\delta_{r',t''}}{E_{r,t}}\prod_{\tilde{t}=0,\tilde{t}\neq t''}^{t'}\left[1+\frac{\delta_{r',\tilde{t}}}{1+g_{r,\tilde{t}}} \right]\notag\\
%                                          &=-\lambda_{r,t}^{-1}\sum_{t'=t}^{\T}\sum_{r'}\lambda_{r',t'}Y_{r',t'}\underbrace{\prod_{\tilde{t}=0}^{t'}\left[1+\frac{\delta_{r',\tilde{t}}}{1+g_{r,\tilde{t}}}\right]}_{=D_{r',t'}}\sum_{t''=t}^{t'}\underbrace{(1+g_{r',t''}+\delta_{r',t''})^{-1}\pf{\delta_{r',t''}}{T_{r',t''}}}_{\equiv -\Theta_{r',t''}}\underbrace{\pf{T_{r',t''}}{E_{r,t}}}_{\equiv\Delta T_{r',t'',r,t}}\notag\\
% &= \sum_{t'=t}^{\T}\sum_{r'}\Phi_{r',t',t}Y_{r',t'}D_{r',t'}\sum_{t''=t}^{t'}\Theta_{r',t''}\Delta T_{r',t'',r,t},  
\end{align}
\begin{align}
  p_{r,t}&=    \sum_{r'}\left[\sum_{t'=t}^{\T}\Phi^K_{r',t',t}\left[Y^0_{r',t'}+(1-\delta_K)K^0_{r',t'}\right]\DK_{r',t'}\sum_{\tilde{t}=t}^{t'}\Theta^{K}_{r',\tilde{t}}\Delta T_{r',\tilde{t},r,t}\right.\\ &\qquad\,\left.-\sum_{t'=t}^{\T-1}\Phi^K_{r',t'+1,t}K^0_{r',t'}\DK_{r',t'+1}\sum_{\tilde{t}=t+1}^{t'-1}\Theta^{K}_{r',\tilde{t}}\Delta T_{r',\tilde{t},r,t}\right]\notag                                        
\end{align}


where we have introduced the marginal change of the capital growth rate with temperature
\begin{equation}
  \label{eq:ThetaK}
  \Theta^K_{r,t}\equiv - (1+\gKN_{r,t}+\dK_{r,t})^{-1}\pf{\dK_{r,t}}{T_{r,t}}.
\end{equation}
\begin{itemize}
\item \todo{Damages are not explicitly time dependent anymore, only indirect dependence via the regional temperature}
\item \todo{May be simplified by assuming that damages $\DK$, marginal changes $\Theta^K$, and $\Delta T$ vary slow compared to the timescale $t$, i.e., we may write $\DK_{r,t}\approx \DK_{r,t+1}$ etc. This should actually be a good approximation because all three terms are not directly time dependent and vary only on the slow timescale of the regional temperature.}
\end{itemize}


% \section{Empirical analysis}
% \label{sec:empAna}
% Following \citet{BAK19} and \citet{HSI14}, we use panel regressions of the form
% \begin{equation}
%   \label{eq:PRA}
%   \ln\big(A_{j,t}\big) = \gamma_j + \delta_{t} + \theta_j t + \sum_{l=0}^{L}\bA_{l}E_{j,t-l}+\epsilon_{j,t}
% \end{equation}
% to capture the effects of climate extremes on county-level TFP $A_{j,t}$, where index $j$ labels the countries, and all time-dependent observables are assumed to vary on the fast (discrete) timescale $t$ of the bio-physical impacts (e.g, annual or monthly).
% In the above equation, the timeseries of TFP $A_{j,t}$ are de-trended log-linearly by accounting for country and year fixed-effects denoted by $\gamma_j$, and $\delta_t$, respectively, as well as for country-specific time trends $\theta_j t$. The persistent effects of climate extremes on total factor productivity are measured by the coefficients $\beta_l$, where the index $l\in[0,\ldots,L]$ numbers the time-lags up to the maximum lag-time $L$. Further, the $E_{j,t-l}$'s denote the bio-physical hazard indicators (e.g, flooded area), and standard errors are denoted by $\epsilon_{j,t}$.
% \begin{itemize}
% \item For the statistics, it might be beneficial to use the same hazard indicator (e.g., affected people) for all impact categories. If this is not possible, we will have to sum the impacts on TFP across categories
%   \item Is there an issue with the units of the coefficients $\beta_l$ if we use different hazard indicators for the impact categories?
% \end{itemize}

% \section{Cumulative country level impacts}
% From Eq.~\eqref{eq:PRA}, we may derive, the cumulative damages at time $t$
% for each realization of hazard indicators. For that, we first define the damage that is caused at time $t$ in country $j$ by a series of hazards $\lbrace E_{j,t-m}^i\rbrace_{m=0}^L$ striking at timepoints  $t-L,\ldots,t$\footnote{\tc{red}{This is not a good description of this term.}} as   
% \begin{equation}
%   \label{eq:dA}
%   \dA_{j,t}\left(\lbrace E_{j,t-m}^i\rbrace_{m=0}^L\right)\equiv \sum_{m=0}^L\beta_m\sum_{i=0}^IE_{j,t-m}^i,
% \end{equation}
% where index $i$ numbers the events that affect country $j$ at time $t$. Accounting for the persistence of damages, the cumulative damage to TFP at time $t\in[2005,2100]$ reads
% \begin{equation}
%   \label{eq:dA}
%   \DA_{j,t,d}\left(\lbrace E\rbrace_{j,b,t}\right)\equiv 1 - \ \prod_{n=0}^{t}\left(1-\dA_{j,t-n}\right),
% \end{equation}
% where $\lbrace E\rbrace_{j,t,d}$ denotes a realization $d\in\mbN_d$ of bio-physical hazards for country $j$ up to time $t$ drawn from the ISIMIP impact ensemble. The ISIMIP2b modeling round includes $1$ SSP (SSP2.6), $4$ GCMs and a various number $\mbN_c$ of global impact models for each event category $c\in\mbC$. For each RCP-SSP-GCM combination, we may obtain a unique temperature path providing a bijective\footnote{\tc{red}{Is this true?}} mapping from time $t$ to global mean temperature $T\equiv T(t)$. For each of these temperature paths, we may then derive $\mbN_d\equiv \prod_{c\in\mbC} \mbN_c$ realization of bio-physical hazards from the  ISIMIP simulations.

% \section{Temperature dependence of impacts}
% Let  $\Delta T\equiv T(t_f)-T(t_0)$ with $t_f>t_0$ denote the interval of global mean temperature of interest. We then subdivide $\Delta T$ into $N_b$ equidistant bins $\Delta T_b$ of width $\Delta T/N_b$. Since the mapping from time to global mean temperature is bijective, we obtain for each temperature bin and each realization $\lbrace E\rbrace_{j,t,d}$ the associated set of cumulative damages as
% \begin{equation}
%   \label{eq:DDA}
%   \mbDA_{j,d}(\Delta T_b)\equiv\lbrace \DA_{j,t,d}\, \vert\, t' \text{ for } T(t')\in\Delta T_b \rbrace.
% \end{equation}
% To obtain relative TFP reductions that can be plugged into the IAM production function (cf.~Eq.~\eqref{eq:CD}), we have to take ensemble averages over (i) the members of the set $\mbDA_{j,d}(\Delta T_d)$ and (ii) the different realizations of biophysical hazards $\mbN_d$,
% \begin{equation}
%   \label{eq:avDA}
%   \langle \DA_{j}\rangle(\Delta T_b)\equiv\big\langle\langle \DA_{j,t,d}\rangle_{\mbN_D}\big\rangle_{\mbDA_{j}}(\Delta T_b).
% \end{equation}
% where $\langle\cdot\rangle$ denotes the ensemble average.

% \begin{itemize}
% \item Does the order in which we take the averages matter?
% \end{itemize}

% \section{Regional integration}
% \label{sec:regInt}
% Since most IAMs operate on the level of world regions $r\in\mbN_r$, we have to average Eq.~\eqref{eq:avDA} over the countries $j$ belonging to each region $r$ (weighted by average population or GDP or asset stock\footnote{\tc{red}{To be discussed!}})  in order to obtain damages $\langle \DA_{r}\rangle(\Delta T_b)$ for the time period associated with $\Delta T_b$.



\section*{Bibliography}
%\bibliography{/home/christian/EXPACT/mendeley_libaries/ownPapers-acclimate_price_model.bib}
\bibliography{lib.bib}

\end{document}
%%% Local Variables: 
%%% mode: latex
%%% TeX-master: t
%%% End: 
