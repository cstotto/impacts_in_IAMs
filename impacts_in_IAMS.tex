%\documentclass[review]{elsarticle} % larger line spacing for more easy review
\documentclass[preprint,3p,authoryear]{elsarticle}

\usepackage{lineno}
\modulolinenumbers[5]

%\journal{Journal of Economic Dynamics \& Control}
\makeatletter
\def\ps@pprintTitle{%
 \let\@oddhead\@empty
 \let\@evenhead\@empty
 \def\@oddfoot{\hfill\today}%
 \let\@evenfoot\@oddfoot}
\makeatother

%% `Elsevier LaTeX' style
% \bibpunct{[}{]}{,}{a}{}{;}
\biboptions{round,semicolon}
\bibliographystyle{elsarticle-harv}
\let\cite\citep
%%%%%%%%%%%%%%%%%%%%%%%

%\smartqed  % flush right qed marks, \eg at endof proof
\clubpenalty = 10000 % Schusterjungen
\widowpenalty = 10000 % Hurenkinder
\displaywidowpenalty = 10000
%Additional packages
\usepackage{lmodern}
\usepackage{supertabular}
\usepackage{amsmath,amssymb}
\usepackage[utf8x]{inputenc}
\usepackage[ngerman,english]{babel}
\usepackage[usenames,dvipsnames]{xcolor}
\usepackage{graphicx} % For includefigure
\usepackage{import}
\usepackage[list=off]{caption}
\usepackage[position=top]{subfig}
%nicer tables -----------------
\usepackage{booktabs}
\newcommand{\ra}[1]{\renewcommand{\arraystretch}{#1}}
\usepackage{enumerate} % roman numbering in enumerations
\usepackage{multirow}
\usepackage{mathtools}
\mathtoolsset{showonlyrefs}
\usepackage{microtype}
\usepackage{rotating}
\usepackage{url}
\usepackage[colorlinks=true,linkcolor=Blue,linktoc=page,citecolor=Blue,urlcolor=BrickRed,breaklinks]{hyperref}
\usepackage{doi}
%\usepackage{breakurl}
\def\UrlBreaks{\do\/\do-\/\do_}
%\renewcommand*{\backref}[1]{}
%\renewcommand*{\backrefalt}[4]{%
%  \ifcase #1 %
%    (Not cited.)%
%  \or
%    (Cited on page~#2.)%
%  \else
%    (Cited on pages~#2.)%
%  \fi}
%\renewcommand*{\backrefsep}{, }
%\renewcommand*{\backreftwosep}{ and~}
%\renewcommand*{\backreflastsep}{ and~}
\newcommand{\RNum}[1]{\uppercase\expandafter{\romannumeral #1\relax}}
\usepackage{xspace} 
\newcommand{\LArrow}{{\Large $\tc{hervor}{\Rightarrow}$~}}
\newcommand{\LASpace}{\\\hspace*{1.6em}}
\providecommand{\herv}[1]{\tc{hervor}{#1}}
\providecommand{\on}[1]{\operatorname{#1}}
\newcommand{\todo}[1]{{\color{red!80!black} TODO: {\itshape\color{orange!70!black} #1}}}
\DeclareRobustCommand{\acclimate}{\emph{acclimate}\xspace}
\DeclareRobustCommand{\Acclimate}{\emph{Acclimate}\xspace}

\usepackage{xspace}  %leerzeichen bei bedarf, aber nicht vor satzeichen
\providecommand{\tc}[2]{\textcolor{#1}{#2}}
\usepackage[]{units}
\providecommand{\myunit}[1]{$\,$\unit{#1}}
\providecommand{\myunitfrac}[2]{$\,\frac{\unit{#1}}{\unit{#2}}$}

%Mathoperators
\newcommand{\E}{\mathcal{E}}
\newcommand{\T}{\mathcal{T}}
\renewcommand{\P}{\mathcal{P}}
\newcommand{\Var}{\operatorname{Var}}
\newcommand{\DDt}{\operatorname{D_{\Delta t}}}
\newcommand{\tDDt}{\operatorname{\tilde{D}_{\Delta t}}}

\newcommand{\Cov}{\operatorname{Var}}
\newcommand{\mbR}{\mathbb{R}}
\newcommand{\mcF}{\mathcal{F}}
%\newcommand{\argmin}{\operatornamewithlimits{argmin}}
\newcommand{\argmax}{\operatornamewithlimits{argmax}}
\renewcommand{\minof}[2]{\min\left[#1\,,\,#2\right]}
\newcommand{\minofthree}[3]{\min\left[#1\,,\,#2\,,\,#3\right]}
\renewcommand{\maxof}[2]{\max\left[#1\,,\,#2\right]}
\newcommand{\pf}[2]{\frac{\partial#1}{\partial#2}}


\newcommand{\first}{1^{\operatorname{st}}}
\newcommand{\secnd}{2^{\operatorname{nd}}}
\newcommand{\third}{3^{\operatorname{rd}}}
\newcommand{\nth}{\text{n}^{\operatorname{th}}}
%\newcommand{\hPSD}{\hyperlink{PSD}{PSD}\xspace}

%##########################
% Real and imaginary parts
%##########################
\renewcommand{\Re}{\operatorname{Re}}
\renewcommand{\Im}{\operatorname{Im}}
%#############################
% Quantities
%#############################
\newcommand{\DA}{D^A}
\newcommand{\DK}{D^K}
\newcommand{\DKA}{D^{K,A}}
\newcommand{\DL}{D^L}
\newcommand{\LA}{L^A}
\newcommand{\bA}{\beta^A}
\newcommand{\dA}{\delta^A}
\newcommand{\dK}{\delta^K}
\newcommand{\GDP}{\operatorname{GDP}}
\newcommand{\gN}{g^{0}}
\newcommand{\gK}{g^{K}}
\newcommand{\AN}{A^{0}}
\newcommand{\tdK}{\tilde{\delta}^K}
\newcommand{\tdA}{\tilde{\delta}^A}

%#############################
% SCC
%#############################
\newcommand{\SCC}{\operatorname{SCC}}
\newcommand{\Lglob}{\mathcal{L}^{\operatorname{glob}}}
\newcommand{\Lreg}{\mathcal{L}^{\operatorname{reg}}}
\newcommand{\mcL}{\mathcal{L}}



\newcommand{\newleftside}{\ifthenelse{\isodd{\thepage}}{\newpage}{\newpage\phantom{placeholder}\thispagestyle{empty} \newpage}}
\newcommand{\newrightside}{\ifthenelse{\isodd{\thepage}}{\newpage\phantom{placeholder}\thispagestyle{empty} \newpage}{\newpage}}
%
\DeclareCaptionLabelFormat{cont}{#1~#2~(cont.)}

\newcommand{\define}{:=}
\newcommand{\set}{:=}
\newcommand{\from}{\leftarrow}
\renewcommand{\to}{\rightarrow}
\newcommand{\bndollar}[1]{\$#1bn}

\makeatletter
\newcommand{\colorwave}[1][blue]{\bgroup \markoverwith{\lower3.5\p@\hbox{\sixly \textcolor{#1}{\char58}}}\ULon}
\font\sixly=lasy6
\makeatother
\newcommand{\todoline}{\colorwave[red]}

\newcommand{\captioncr}{{\protect\\}}
\AtBeginDocument{\renewcommand\appendixname{Appendix~}}
\def\UrlBreaks{\do\/\do-\/\do_}

\makeatletter
% for aligning in math modes:
\newcommand{\pushright}[1]{\ifmeasuring@#1\else\omit\hfill$\displaystyle#1$\fi\ignorespaces}
\newcommand{\pushcenter}[1]{\ifmeasuring@#1\else\omit\hfill$\displaystyle#1$\hfill\fi\ignorespaces}
\newcommand{\pushleft}[1]{\ifmeasuring@#1\else\omit$\displaystyle#1$\hfill\fi\ignorespaces}
% instead of eqnarray use:
% \begin{alignat}{4} a&&\pushcenter{=}&b \end{alignat}
\makeatother

\newcommand{\annot}[2][]{%
  \pdfannot width \linewidth height 2\baselineskip depth 0pt{%
    /Subtype/Text%
    /Open false
    /Name /Comment%
    /CA .4%
    /C [.3 .6 .9]%
    /T (\pdfescapestring{#1})%
    /Contents(\pdfescapestring{\detokenize{#2}})%
  }
}
\ifdefined\note\relax\else
\newcommand\note[1]{\marginnote{\begin{minipage}{0.8\marginparwidth}\footnotesize\raggedright{}#1\end{minipage}}}
\fi

%#############################
% Sets
%#############################
\newcommand{\mbC}{\mathbb{C}}
\newcommand{\mbN}{\mathbb{N}}
\newcommand{\mbDA}{\mathbb{D}^A}
%%%%%%%%%%%%%%%%
% Farben
%%%%%%%%%%%%%%%%
\definecolor{dunkelgrau}{rgb}{0.8,0.8,0.8}
\definecolor{hellgrau}{RGB}{204,216,226}

\AtBeginDocument{\renewcommand\appendixname{Appendix~}}

%\input{latexdiff}

\begin{document}
%\showthe\columnwidth 338
%\date{\today}

\begin{frontmatter}

  \title{How to include climate impacts in macroeconomic growth models?}

\author[PIK]{C.~Otto\corref{mycorrespondingauthor}}
\cortext[mycorrespondingauthor]{Corresponding author}
% \ead{christian.otto@pik-potsdam.de}



\author[PIK]{T.~Geiger}
\author[PIK]{K.~Frieler}


\address[PIK]{Potsdam Institute for Climate Impact Research, Potsdam, Germany}

% \begin{abstract}
% \end{abstract}

% \begin{keyword}
%   disaster impact analysis\sep
%   higher-order effects\sep
%   economic network\sep
%   resilience\sep
%   dynamic input-output model\sep
%   agent-based modeling
% %Q540 Climate; Natural Disasters and Their Management; Global Warming
% %F100 Trade 
% %F180 Trade and Environment
% %Q560 Environment and Development; Environment and Trade; Sustainability; Environmental Accounts and Accounting; Environmental Equity; Population Growth
% %\JEL F100\sep F180\sep Q540\sep Q560
% \end{keyword}


\end{frontmatter}

%\linenumbers

%\newpage

\section{Setting}
\label{sec:set}
The production system in integrated assessment models (IAMs) is often described by an aggregate production function of Cobb-Douglas type
\begin{equation}
  \label{eq:CD}
  Y(\tau,T)\equiv \big(1-\DA(\tau,T)\big)A(\tau)\left[\big(1-\DK(\tau,T)\big)K(\tau) \right]^\alpha L(\tau)^{\alpha-1},
\end{equation}
where $\tau$ is the slow timescale of the IAM (e.g, decadal), and $A$,$K$,$L$,$\alpha$ denote total factor productivity (TFP), capital stock, labor, and the Cobb-Douglas coefficient, respectively. Further, $\DA$ and $\DK$ denote climate damages to $A$ and $K$, which are functions of $\tau$ and the global mean temperature $T$.

In the following, we assume (w.l.o.g.) that climate extremes have an immediate impact on $K$, and have persistent impacts on $A$.

\section{Empirical analysis}
\label{sec:empAna}
Following \citet{BAK19} and \citet{HSI14}, we use panel regressions of the form
\begin{equation}
  \label{eq:PRA}
  \ln\big(A_{j,t}\big) = \gamma_j + \delta_{t} + \theta_j t + \sum_{l=0}^{L}\bA_{l}E_{j,t-l}+\epsilon_{j,t}
\end{equation}
to capture the effects of climate extremes on county-level TFP $A_{j,t}$, where index $j$ labels the countries, and all time-dependent observables are assumed to vary on the fast (discrete) timescale $t$ of the bio-physical impacts (e.g, annual or monthly).
In the above equation, the timeseries of TFP $A_{j,t}$ are de-trended log-linearly by accounting for country and year fixed-effects denoted by $\gamma_j$, and $\delta_t$, respectively, as well as for country-specific time trends $\theta_j t$. The persistent effects of climate extremes on total factor productivity are measured by the coefficients $\beta_l$, where the index $l\in[0,\ldots,L]$ numbers the time-lags up to the maximum lag-time $L$. Further, the $E_{j,t-l}$'s denote the bio-physical hazard indicators (e.g, flooded area), and standard errors are denoted by $\epsilon_{j,t}$.
\begin{itemize}
\item For the statistics, it might be beneficial to use the same hazard indicator (e.g., affected people) for all impact categories. If this is not possible, we will have to sum the impacts on TFP across categories
  \item Is there an issue with the units of the coefficients $\beta_l$ if we use different hazard indicators for the impact categories?
\end{itemize}

\section{Cumulative country level impacts}
From Eq.~\eqref{eq:PRA}, we may derive, the cumulative damages at time $t$
for each realization of hazard indicators. For that, we first define the damage that is caused at time $t$ in country $j$ by a series of hazards $\lbrace E_{j,t-m}^i\rbrace_{m=0}^L$ striking at timepoints  $t-L,\ldots,t$\footnote{\tc{red}{This is not a good description of this term.}} as   
\begin{equation}
  \label{eq:dA}
  \dA_{j,t}\left(\lbrace E_{j,t-m}^i\rbrace_{m=0}^L\right)\equiv \sum_{m=0}^L\beta_m\sum_{i=0}^IE_{j,t-m}^i,
\end{equation}
where index $i$ numbers the events that affect country $j$ at time $t$. Accounting for the persistence of damages, the cumulative damage to TFP at time $t\in[2005,2100]$ reads
\begin{equation}
  \label{eq:dA}
  \DA_{j,t,d}\left(\lbrace E\rbrace_{j,b,t}\right)\equiv 1 - \ \prod_{n=0}^{t}\left(1-\dA_{j,t-n}\right),
\end{equation}
where $\lbrace E\rbrace_{j,t,d}$ denotes a realization $d\in\mbN_d$ of bio-physical hazards for country $j$ up to time $t$ drawn from the ISIMIP impact ensemble. The ISIMIP2b modeling round includes $1$ SSP (SSP2.6), $4$ GCMs and a various number $\mbN_c$ of global impact models for each event category $c\in\mbC$. For each RCP-SSP-GCM combination, we may obtain a unique temperature path providing a bijective\footnote{\tc{red}{Is this true?}} mapping from time $t$ to global mean temperature $T\equiv T(t)$. For each of these temperature paths, we may then derive $\mbN_d\equiv \prod_{c\in\mbC} \mbN_c$ realization of bio-physical hazards from the  ISIMIP simulations.

\section{Temperature dependence of impacts}
Let  $\Delta T\equiv T(t_f)-T(t_0)$ with $t_f>t_0$ denote the interval of global mean temperature of interest. We then subdivide $\Delta T$ into $N_b$ equidistant bins $\Delta T_b$ of width $\Delta T/N_b$. Since the mapping from time to global mean temperature is bijective, we obtain for each temperature bin and each realization $\lbrace E\rbrace_{j,t,d}$ the associated set of cumulative damages as
\begin{equation}
  \label{eq:DDA}
  \mbDA_{j,d}(\Delta T_b)\equiv\lbrace \DA_{j,t,d}\, \vert\, t' \text{ for } T(t')\in\Delta T_b \rbrace.
\end{equation}
To obtain relative TFP reductions that can be plugged into the IAM production function (cf.~Eq.~\eqref{eq:CD}), we have to take ensemble averages over (i) the members of the set $\mbDA_{j,d}(\Delta T_d)$ and (ii) the different realizations of biophysical hazards $\mbN_d$,
\begin{equation}
  \label{eq:avDA}
  \langle \DA_{j}\rangle(\Delta T_b)\equiv\big\langle\langle \DA_{j,t,d}\rangle_{\mbN_D}\big\rangle_{\mbDA_{j}}(\Delta T_b).
\end{equation}
where $\langle\cdot\rangle$ denotes the ensemble average.

\begin{itemize}
\item Does the order in which we take the averages matter?
\end{itemize}

\section{Regional integration}
\label{sec:regInt}
Since most IAMs operate on the level of world regions $r\in\mbN_r$, we have to average Eq.~\eqref{eq:avDA} over the countries $j$ belonging to each region $r$ (weighted by average population or GDP or asset stock\footnote{\tc{red}{To be discussed!}})  in order to obtain damages $\langle \DA_{r}\rangle(\Delta T_b)$ for the time period associated with $\Delta T_b$.


%\section*{Acknowledgments}

\section*{Bibliography}
%\bibliography{/home/christian/EXPACT/mendeley_libaries/ownPapers-acclimate_price_model.bib}
\bibliography{lib.bib}

\end{document}
%%% Local Variables: 
%%% mode: latex
%%% TeX-master: t
%%% End: 
